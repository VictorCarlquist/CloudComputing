\documentclass[xcolor=dvipsnames]{beamer} 
\usecolortheme[named=Brown]{structure} 
\usetheme[height=7mm]{Rochester} 

\usepackage[brazilian]{babel}
\usepackage[T1]{fontenc} 
\usepackage{graphicx}

\author{Denis F. de Carvalho,
	Guilherme A. de Macedo,
	Matheus L. Domingues da Silva e 
	Victor H. Carlquist da Silva}

\title{Could Computing}
		
\hypersetup{pdfpagemode=FullScreen}

\begin {document}

\frame { \maketitle }

\frame {
	\frametitle {Introdução}
	\begin{itemize}
		\item Objetivo;
		\item Apresentação dos conceitos básicos;
		\item Soluções disponíveis no mercado;
		\item Conclusão.
	\end{itemize}
	
}

\frame {
	\frametitle {O que é Cloud Computing?}
	\begin{itemize}
		\item Espaço;
		\item Economia;
		\item Tempo;
		\item Espaço;
		\item Backup;
		\item Configuração.
	\end{itemize}
	
}

\frame {
	\frametitle {Por que surgiu?}
	\begin{itemize}
		\item Disponibilidade de serviço;
		\item Despreocupação com a tecnologia;
		\item Despreocupação com a infraestrutura.
	\end{itemize}
	
}

\frame {
	\frametitle {Desvantagens}
	\begin{itemize}
		\item Lentidão por armazenar os arquivos na nuvem;
		\item Conferir a tercerios informações estratégicas da empresa.
	\end{itemize}
	
}

\frame {
	\frametitle {Modelos de Computação nas Nuvens}
	\begin{itemize}
		\item SaaS;
		\item PaaS;
		\item IaaS.
	\end{itemize}
	
}

\frame {
	\frametitle {Possibilidades - Soluções Disponíveis no Mercado}
	\begin{itemize}
		\item Gmail (SaaS);
		\item Heroku (PaaS);
		\item OpenShift (PaaS);
		\item IaaS (PaaS);
		\item Amazon Web Service (IaaS).
	\end{itemize}
	
}

\frame {
	\frametitle {Conclusão}
	Cloud Computing é um conceito que está se tornando cada vez mais popular, embora não seja a solução para todos os problemas do mercado, ela garante uma economia muito grande para as empresas já que serviços dedicados poderiam ser criados atravéz de poucos cliques.
}

\end{document}
