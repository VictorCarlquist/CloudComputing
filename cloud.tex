\documentclass{abnt}

\input{Base}

%%%%% Dados para criação da capa e folha de rosto %%%%
\autor{	Denis F. de Carvalho,
	Guilherme A. de Macedo,
	Matheus L. Domingues da Silva e 
	Victor H. Carlquist da Silva
}
\titulo{Could Computing}
\orientador{Avelino Natal Bazanella Junior}
\comentario{Trabalho apresentado ao Prof. Avelino Bazanela Junior, na disciplina de Redes de Computadores presente no $2^{a}$ modulo do curso de Tecnologia em Análise e Desenvolvimento de Sistemas no IFSP-CJO.}
\instituicao{Instituto Federal de Educação, Ciência e Tecnologia de São Paulo -- \textit{campus} Campos do Jordão}
\local{Campos do Jordão}
\data{\today}

\begin{document}

	% Para utilizar o formato padrão de capa da ABNT, substituí o comando \maketitle pelo comando \capa.
	\capa
	
	\folhaderosto
	
	\begin{resumo}
		Este trabalho tem por objetivo mostrar e explicar o funcionamento da tecnologia de computação em nuvem (cloud computing). 
		A construção desse trabalho foi baseada em pesquisas em \textit{sites}, gráficos e tabelas, bem como a consulta de livros especializados.
	\end{resumo}
	
	\sumario
	
	\listadetabelas
	
	\listadefiguras
	
	\chapter{Introdução}
		teste
	\chapter{O que é \textit{Cloud Computing} ?}
		teste
	\chapter{Por que surgiu?}
		teste
	\chapter{Modelos de Computação nas Nuvens}

	Com a utilização da computação em nuvem é possível oferecer o 
	hardware e o software como serviços. 
	Entre os diversos recursos disponiveis com a utilização dessa tecnologia, 
	destacam-se os metodos de SaaS, PaaS e IaaS.

	\section{SaaS}
	
	O Software como serviço (\textit{Software as a Service} (SaaS)) é uma forma 
	de comercialização do software pela internet. Nesse modelo, o fornecedor 
	fica responsável pela instalação, configuração e disponibilização do 
	software, e o usuário apenas paga pelo uso.
	Geralmente o acesso do usuário ao sistema é feito pela interface 
	de um navegador \textit{web}.

	\begin{figure}[h]
		\centering
		\includegraphics[width=10cm, keepaspectratio]{img/SaaS.png}
		\caption{Modelo SaaS.}
		\label{saas}
	\end{figure}
	
	Este modelo possibilita maior flexibilidade ao usuário, pois este não 
	precisa se preocupar como é feita a configuração do sistema em uso. Cabe 
	ao usuário apenas utilizar o serviço disponível.

	\section{PaaS}
		A Plataforma como Serviço (\textit{Platform as a Service} (PaaS)) possibilita a escolha rápida de recursos para desenvolvimento de aplicações.
		
		Esta plataforma é considerada a mais confusa das camadas do \textit{cloud}, geralmente sendo confundida com o SaaS ou IaaS (\textit{Infrastructure as a Service}).
		
		Uma plataforma na computação, se referindo ao \textit{software}, pode ser definida como os Sistemas Operacionais, por exemplo, Windows\texttrademark , Linux e Mac OS, e pode ser definida como os \textit{frameworks} para os aplicativos. Os SGBDs (Sistema de Gerenciamento de Banco de Dados) também estãos nesta camada.

	\begin{figure}[h]
		\centering
		\includegraphics[width=8cm, keepaspectratio]{img/figure1.jpg}
		\label{fig_paas}
		\caption{Estrutura do PaaS}
	\end{figure}

	O que um bom provedor PaaS precisa ter:
	
	\begin{itemize}
		\item Estrutura de desenvolvimento de aplicativo: Uma estrutura de desenvolvimento de aplicativo robusta desenvolvida em tecnologia amplamente usada, por exemplo, o Java;
            	\item Disponibilidade: A plataforma de opção deve estar acessível e disponível em qualquer lugar, a qualquer hora;
            	\item Escalabilidade: A plataforma deve ser inteligente o suficiente para aproveitar a capacidade elástica de uma infraestrutura;
            	\item Segurança: Deve possuir dispositivos contra ataques;
            	\item Inclusão: A plataforma deve fornecer a capacidade de incluir, embarcar e integrar outros aplicativos desenvolvidos nas mesmas plataformas ou em outras;
            	\item Portabilidade: A plataforma deve permitir que as empresas movam o aplicativo de uma IaaS para outra.
	\end{itemize}

	\section{IaaS}
		Infraestrutura como Serviço (\textit{Infrastructure as a Service} (IaaS)) é a camada do \textit{Cloud Computer} de mais baixo nível. Ela pode ser defida como sendo a 'capacidade compucional' da nuvem. Ela é responsável pela infraestrura, ou seja, é nesta camada que se define a quantidade de processamento, de armazenamento, de memória RAM, etc. Toda esta estrutura pode ser encontrada em nossas casas, mas em escala muito menor. O IaaS trabalha nesse nicho, mas em escala industrial.
		
		O IaaS não é constituído por PCs(\textit{Personal Computer}), mas por diversos servidores robustos, e os dados ficam em \textit{storages}, que são máquinas que possuem grande contingência, poder de armazenamento e velocidade.
		
		Hoje em dia existem diversos serviços, que com apenas um clique pode se criar um servidor com a configuração que se deseja.
		
		O IaaS fornece seus serviços as outras duas camadas superiores, o PaaS e o SaaS.
		 
	\chapter{Possibilidades -- Soluções disponíveis no mercado}
	  \section{Apple iCloud - SaaS}
	\chapter{Conclusão}
		teste
\end{document}
