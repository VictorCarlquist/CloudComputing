\documentclass{abnt}

\input{Base}

%%%%% Dados para criação da capa e folha de rosto %%%%
\autor{	Denis F. de Carvalho,
		Guilherme A. de Macedo,
		Matheus L. Domingues da Silva e 
		Victor H. Carlquist da Silva
}
\titulo{Could Computing}
\orientador{Avelino Natal Bazanella Junior}
\comentario{Trabalho apresentado ao Prof. Avelino Bazanela Junior, na disciplina de Redes de Computadores
			presente no $2^{a}$ modulo do curso de Tecnologia em Análise e Desenvolvimento de Sistemas no IFSP-CJO.}
\instituicao{Instituto Federal de Educação, Ciência e Tecnologia de São Paulo -- \textit{campus} Campos do Jordão}
\local{Campos do Jordão}
\data{\today}

\begin{document}

	% Para utilizar o formato padrão de capa da ABNT, substituí o comando \maketitle pelo comando \capa.
	\capa
	
	\folhaderosto
	
	\begin{resumo}
		Este trabalho tem por objetivo mostrar e explicar o funcionamento da tecnologia de computação em nuvem (cloud computing). 
		A construção desse trabalho foi baseada em pesquisas em \textit{sites}, gráficos e tabelas, bem como a consulta de livros especializados.
	\end{resumo}

	\begin{abstract}
		This work aims to show and explain the workings of the Six Sigma program. The construction of this work was based on research on sites, graphs and tables, and consultation of specialized books.
	\end{abstract}
	
	\sumario
	
	\listadetabelas
	
	\listadefiguras
	
	\newpage
	
	\section{Introdução}
		teste
	\section{O que é \textit{Cloud Computing} ?}
		teste
	\section{Por que surgiu?}
		teste
	\section{Modelos de Computação nas Nuvens}
		teste
	\subsection{SaaS}
		teste
	\subsection{PaaS}
		teste
	\subsection{IaaS}
		Infraestrutura como Serviço (\textit{Infrastructure as a Service} (IaaS)) é a camada do \textit{Cloud Computer} de mais baixo nível. Ela pode ser defida como sendo a 'capacidade compucional' da nuvem. Ela é responsável pela infraestrura, ou seja, é nesta camada que se define a quantidade de processamento, de armazenamento, de memória RAM, etc. Toda esta estrutura pode ser encontrada em nossas casas, mas em escala muito menor. O IaaS trabalha nesse nicho, mas em escala industrial.
		
		O IaaS não é constituído por PCs(\textit{Personal Computer}), mas por diversos servidores robustos, e os dados ficam em \textit{storages}, que são máquinas que possuem grande contingência, poder de armazenamento e velocidade.
		
		Hoje em dia existem diversos serviços, que com apenas um clique pode se criar um servidor com a configuração que se deseja.
		
		O IaaS fornece seus serviços as outras duas camadas superiores, o PaaS e o SaaS.
		
		
		
		 
		 
	\section{Possibilidades -- Soluções disponíveis no mercado}
		teste
	\section{Conclusão}
		teste
		
\end{document}
